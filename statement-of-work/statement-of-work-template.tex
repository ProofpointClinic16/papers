%%% Clinic Statement of Work Template

%%% !!! HMC STUDENTS SHOULD REMOVE THE FOLLOWING COPYRIGHT NOTICE FROM
%%% !!! FINAL SUBMISSIONS.

%%% Copyright (C) 2004-2015 Department of Mathematics, Harvey Mudd College.
%%%
%%% This file is part of the hmcclinic class document provided to
%%% HMC mathematics students.
%%%
%%% See the COPYING document, which should accompany this
%%% distribution, for information about distribution and
%%% modification of the document and its components.

%%% !!! END COPYRIGHT NOTICE.


%%% Clinic reports use the hmcclinic class, which should be located
%%% somewhere in TeX's search path.

%%% You must set a document-class option to tell the class what
%%% department your report will be for.
\documentclass[cs,proposal]{hmcclinic}
\usepackage{enumitem}

%% Supported department classes are
%%
%%    biology
%%    computer-science
%%    chemistry
%%    engineering
%%    mathematics
%%    physics

%% For your Statement of Work, use "proposal", as set above.


%%% The major difference between the Statement of Work and a midyear
%%% or final report is that the statement of work is typeset as an
%%% article, which means that the highest level of structural
%%% division available to you is \section rather than \chapter.

%%% There are also some changes in pagination styles and content
%%% that reflect the briefer nature of the proposal.  For example,
%%% in the longer reports, you use \frontmatter, \mainmatter, and
%%% \backmatter to separate some sections of the report from
%%% others.  In the statement of work, you don't need those
%%% commands, as no such division is necessary.

%%% Other packages needed by your document may be loaded here.
% \usepackage{url}              % For formatting URLs and other web or
                                % file references.

%%% Provide additional context around errors. 
\setcounter{errorcontextlines}{1000}


%%% Information about this document.

%%% I find it most useful to put identifying information about a
%%% document near the top of the preamble.  Technically, this
%%% information must precede the \maketitle command, which often
%%% appears immediately after the beginning of the document 
%%% environment.  Placing it near the top of the document makes it
%%% easier to identify the document, and keeps it out from getting
%%% mixed up with the real meat of the document.

%%% We use the same set of commands for specifying information about
%%% the people involved with the project that are used in the longer
%%% reports, so you can copy most of this information directly into
%%% your midyear and final reports.

%%% So, some questions.

%% What is the name of the company or organization sponsoring your project?
\sponsor{Proofpoint, Inc.}

%% What is the title of your report?
\title{Predicting Malicious URLs}

%% Who are the authors of the report (your team members)?  (Separate
%% names with \and.)
\author{Vidushi Ojha (Project Manager) \and James Best \and Aidan Cheng \and Kevin Herrerra \and Carli Lessard}

%% What is your faculty advisor's name?  (Again, separate names with
%% \and, if necessary.)
\advisor{Elizabeth Sweedyk}

%% Liaison's name or names?
\liaison{Thomas Lynam \and Mike Morris}

%% Did you have an outside consultant help you with this project?  Put
%% their names in the \consultant command.
%\consultant{Joseph Jones}

%%% End of information section.

%%% New commands and environments.

%%% You can define your own commands and environments here.  If you
%%% have a lot of material here, you might want to consider splitting
%%% the commands and environments into a separate ``style'' file that
%%% you load with \usepackage.

\newcommand{\coolcommand}[1]{#1 is cool.} % Lets everyone know that
                                % the person or thing that you provide
                                % as the argument to the command is
                                % cool.


%%% Some theorem-like command definitions.

%%% The \newtheorem command comes from the amsthm package.  That
%%% package is *not* automatically loaded by the class file, so if
%%% you choose to use these commands, you'll need to specify the
%%% "amsthm" document-class option.

%%% Note that these definitions have changed from the version in the
%%% sample report document by dropping the ``within'' argument.  See
%%% Gratzer's _Math into LaTeX_ or the AMS-LaTeX documentation for
%%% more details.

% \newtheorem{thm}{Theorem}
% \newtheorem{Theo1}{Theorem}
% \newtheorem{Theo2}{Theorem}
% \newtheorem{Lemma}{Lemma}


%%% If you find that some words in your document are being hyphenated
%%% incorrectly, you can specify the correct hyphenation using the
%%% \hyphenation command.  Note that words are separated by
%%% whitespace, as shown below.

\hyphenation{ap-pen-dix wer-ther-i-an}


%%% The start of the document!

%% The document environment is the main environment in any LaTeX
%% document.  It contains other environments, as well as your text.

\begin{document}

%%% In a longer document (such as your midterm and final reports),
%%% you would have separate \frontmatter, \mainmatter, and
%%% \backmatter commands to define some large chunks of your
%%% document.  For the Statement of Work, which is a short document,
%%% we don't need these commands.

%%% Your Statement of Work begins with a title page.  The title page
%%% is formatted by commands in the document class file, so you
%%% don't need to worry about what it looks like -- just putting the
%%% \maketitle command in your document (and filling in the necessary
%%% information for the identification commands above) is enough.
\maketitle

\tableofcontents

%%% In a longer document or an article being submitted to a journal
%%% or conference, you would probably have an abstract that
%%% summarized the purpose of the document.  We don't need that for
%%% a Statement of Work.

%%% Similarly, in longer documents you would probably have commands
%%% to include a table of contents and lists of figures or tables.
%%% For a short document such as the Statement of Work, we don't
%%% need these commands.


%%% Content.

%%% For smaller documents---especially those you're writing by
%%% yourself---you might write your entire report using a single LaTeX
%%% source file.  For larger documents, we recommend that you split
%%% the source file into several separate, smaller files.  The smaller
%%% files are ``included'' into your main, or ``master'' document
%%% using \include commands.  See the template file for the Clinic
%%% reports for more details on how to split a LaTeX project into
%%% smaller files.


%%% The body of your Statement of Work should appear here.  See
%%% Chapter 4 in _The Mathematics Clinic in Brief: A Handbook_ for
%%% more details on what you should include in a Statement of Work.

\newpage

%%%

\section{Project Motivation}

Proofpoint is a cybersecurity company that provides security and data protection solutions to other companies. Amongst their many products, they provide an inbound email URL screening service that scans URLs embedded in clients' emails, and determines whether or not they lead to sites containing malware.
\\\\
Determining the maliciousness of URLs is a critical component of Proofpoint's security suite because of the ease with which malware can affect clients' machines. Malware can covertly install itself when a user clicks on a URL, and compromise personal and sensitive information without the victim knowing it. Indeed, attackers can send emails containing these malicious URLs from seemingly benign sources, like someone in the victim's address book, making such emails hard to detect for the client. Thus, Proofpoint would like to to block URLs before they even get clicked.
\\\\
Proofpoint's solution currently redirects every URL embedded in an email through their servers, where they employ a filter to distinguish between URLs that should and should not be blocked. Their current filtration technique has approximately 70\% accuracy in determining whether a URL is malicious. This method checks how many times the URL appears in a certain time period and context, and how many domains it goes through.\footnote{These heuristics are useful because they provide characteristics common to malicious URLs. Many will be sent through multiple domains to try to hide where they came from, and they will be sent multiple times to try to get through to the client.} If a URL reaches a certain threshold with regards to these two test metrics, it will be sent to Proofpoint's \textit{sandboxing environment} for further testing. Sandboxing, the practice of opening a URL on a virtual machine and simulating its effects, is currently the most accurate method of determining whether a URL is malicious. If the sandbox becomes infected with malware, Proofpoint will block that URL in the future.
\\\\
However, sandboxing is slow, which makes it expensive in both time and money. Given the billions of emails Proofpoint's security suite sees every day, it is unfeasible to sandbox every single one. It would therefore be useful to have a more effective way of determining which URLs are malicious, as this would significantly reduce the number of URLs that have to go through the expensive sandboxing process.
\\\\
Although their current method of predicting malicious URLs is relatively effective, there is much room for improvement. Proofpoint is interested in improving the number of URLs blocked overall, but also the number of URLs blocked before the client has a chance to click on them. Our team believes that methods of machine learning are well suited for this problem: given the vast amounts of data, classification and pattern matching are exactly the kinds of solutions needed for this problem. Our hypothesis is that there are common characteristics shared by malicious URLs, and indeed, others before us have investigated such characteristics (see section \ref{feature-selection}). An effective learning technique could determine these characteristics and use them as facets of a learning model. For this reason, we plan on employing a number of different machine learning techniques to the malicious URL detection problem.

%%%

\section{Problem Statement}

Proofpoint processes billions of URLs a day to determine if they are malicious. However, their sandboxing method is much too expensive a process for it to be attempted for every URL they see. Thus, there is a need for better predictive model that reduces the number of expensive sandbox tests that must be performed. This system should learn from existing metadata about URLs. The ideal solution for this problem would be able to learn from its predictions. For example, if it predicts a URL to be malicious, and that URL is deemed safe by the sandboxing environment, the predictor should refine its model to account for this data. The problem, then, is to construct a model with these characteristics that can make these predictions for the vast number of URLs being processed by Proofpoint on a daily basis.

%%%

\section{Goals}

Over the course of this academic year, our team intends to design and implement a system that uses machine learning to detect malicious URLs before they are clicked. The primary aim of this systems is to block these URLs before Proofpoint's clients view them, thereby avoiding any opportunity for the URL to be clicked. However, while the primary goal is to block URLs before they are clicked, there are three metrics total that will be measured in order to evaluate the success of our model. These are:

\begin{itemize} \itemsep0em
\item The proportion of malicious URLs tagged as malicious before they are clicked
\item The proportion of malicious URLs tagged as malicious total
\item The proportion of malicious URLs submitted to the sandbox for additional analysis
\end{itemize}

We are aiming to improve upon the current proportion of URLs blocked before clicking, which is currently at around 70\%. Although we have no precise numbers to outperform with regards to the other two metrics, our aim is to accurately block as many URLs as possible while minimizing inaccurate predictions.
\\\\
The system we aim to build must have the following features:

\begin{enumerate} \itemsep0em
\item For any given input URL that is given to it, the system returns a score between 0 and 1, where the score indicates the probability of the URL being malicious.
\item Using the above score, each sample will be classified as either dangerous or not, based on some cutoff. For instance, if we decide on a cutoff of 0.7, then anything with a score of 0.7 or above will be considered dangerous.
\item Our model will explain how the score was assigned, for instance by pointing to characteristics of the URL that make it more likely that it is malicious.
\end{enumerate}

%%%

\section{Classifier Types}

The following is a summary of the classifiers our team will investigate over the course of this project.

\subsection{Naive Bayes}

\subsection{Support Vector Machines}

\subsection{Clustering}

%%%

\section{Feature Selection} \label{feature-selection}

Stuff

%%%

\section{Architecture}

Stuff

%%%

\section{Deployment Strategy}

Stuff

%%%

\section{Schedule}

\subsection{Phases Overview}

\subsection{Workflow and Deliverables}

%%%

\section{Tools}

Stuff

%%%

\section{References}

Stuff

\newpage



%%% Appendices.

%%% For your Statement of Work, you probably won't have any
%%% appendices, but you could include some if you really needed to.

%%% The appendices are delineated with the \appendix command.
%%% Individual appendices are begun with the standard \chapter or
%%% \section commands.  In our example, we'll \include them just as we
%%% did other chapters.

%%% Even in a relatively short document such as your statement of
%%% work, you might need to have appendices.  If so, uncomment
%%% the \appendix command and add them below (remember, the
%%% top-level structural command in this format is section).

% \appendix

%%% Bibliography.

%%% BibTeX is the tool to use for citations and layout of your
%%% bibliography.  Instead of having to type ``[5]'' or ``(Jones,
%%% 1968)'' (and keep track of which citation is which and renumber
%%% them as you add more references to your bibilography), you use
%%% special commands that allow BibTeX and LaTeX to automatically put
%%% the correct information in the right place.

%%% Section 5.6 in _The Mathematics Clinic in Brief: A Handbook_,
%%% talks about using BibTeX to format your bibliography and
%%% citations.

%%% Depending on your field, it may or may not be appropriate to list
%%% references for which you haven't included specific citations.  If
%%% your field sanctions such practices, or if you just want to get an
%%% idea of what you have in your bibliography file, you can include
%%% everything with the \nocite{*} command.
\nocite{*} 


%%% The appearance of your bibliography and citations in your text are
%%% defined by a combination of any bibliography-related LaTeX
%%% packages (such as natbib, harvard, or chicago) and the particular
%%% bibliography style file that you load with the \bibliographystyle
%%% command.  Bibliography-style files end in .bst; you can find them
%%% by searching your file system using whatever tools you have for
%%% doing searches.  (On most modern Unices, ``locate .bst'' will give
%%% you an idea of what's available.)

\bibliographystyle{hmcmath}

%%% The particular bibliography data file or files that you want to
%%% use are specified with the \bibliography file.  Multiple files are
%%% separated by commas.

%%% You might want to use multiple bibliography (or ``bib'') files if
%%% you had a master bib file containing references you use again and
%%% again, and another containing only records for references for a
%%% particular project.

%%% Many people create a single, large bib file that they use for
%%% everything they write.  That approach requires you to \cite every
%%% reference that you want to use in your document -- using
%%% \nocite{*} with a huge bibliography database will give you a large
%%% bibliography containing many references you haven't consulted for
%%% your particular document!

\bibliography{sample}


%%% Glossary or Index.

%%% Having a glossary or index in a statement of work is overkill.
%%% Just define your terms in the text and you'll be fine.

\end{document}
